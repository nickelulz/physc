\documentclass[12pt]{article}
\usepackage{enumitem}
\usepackage{amsmath}

\newcommand{\dd}[1]{\mathrm{d}#1}

\setlength\parindent{0pt}

\begin{document}

{\large PHYS C 143 UNIFORM ELECTRIC FIELDS}

\begin{enumerate}

  \item A charge of $15\mu C$ ($1.5x10^{-5}$) is placed in a uniform electric field which has a field strength of $E=88,000$ N/C ($8.8x10^4$ N/C);
\begin{enumerate}
  \item What will be the magnitude of the electrostatic force acting on this charge?
    \begin{align}
      E = \frac{F_{e}}{q} \\
      F_{E} = Eq \\
      F_{E} = 8.8 \times 10^{4} \cdot 1.5 \times 10^{-5} \\
      F_{E} = 1.32 \mathrm{N}
    \end{align}

  \item How much work would be done in moving this charge a distance of 135cm (1.35 m) against the electric field?
    \begin{align}
      W = \int F \dd r = F_{E} \cdot r \\
      W = 1.32 \cdot 1.35 \\
      W = 1.782 \mathrm{Nm}
    \end{align}

  \item What will be the potential difference between these two points?
    \begin{align}
      \Delta V = E\Delta x \\
      \Delta V = 8.8\times 10^4 \cdot 1.35 \\
      \Delta V = 1.188\times 10^5 \mathrm{V}
    \end{align}
\end{enumerate}

\item A proton [$m_{p} = 1.67 \times 10^{-27} kg, q_{p} = 1.6 \times 10^{-19} C$] is placed at point A in a uniform electric field which has a field strength of $E = 4500$ N/C ($4.5\times 10^{3}$) and which is directed toward the top of the page as shown in the diagram to the right.
\begin{enumerate}
  \item What will be the direction of the electrostatic force acting on this proton while at point A? \textit{Up.}

  \item What will be the magnitude of the electrostatic force acting on this proton while at point A?
    \begin{align}
      F_{E} = Eq \\
      F_{E} = 4.5 \times 10^3 \cdot 1.6 \times 10^{-19} \\
      F_{E} = 7.2 \times 10^{-16} \mathrm{N}
    \end{align}

  \item How much work will be done in moving this charge a distance of $12.0cm$ ($0.12$ m) against this electric field to point B?
    \begin{align}
      W = \int F \dd \Delta x = F \Delta x \\
      W = 7.2 \times 10^{-16} \cdot 0.12 \\
      W = 8.64 \times 10^{-17} \mathrm{J}
    \end{align}

  \item How will the electrostatic potential at point B in this field compare with the electrostatic potential at point A?
    \begin{align}
      \Delta V_{A \to B} = E \Delta x_{A \to B} \\
      \Delta V_{A \to B} = 4.5 \times 10^{3} \cdot 0.12\\
      \Delta V_{A \to B} = 540 \mathrm{V}
    \end{align}

  \item What will be the potential difference between points A and B? Suppose that this proton is then released and is allowed to accelerate back to point A.
    \textit{The same as (d).}
    \begin{align}
      \Delta V_{A \to B} = 540 \mathrm{V}
    \end{align}

  \item What will be the velocity of this proton when it returns to point A?
    \begin{align}
      E_{0} = E_{f} \\
      U_{E} = K_{f} \\
      q_{p}V_{0} = \frac{1}{2}m_{p}v_{f}^{2} \\
      v_{f} = \sqrt{\frac{2q_{p}V_{0}}{m_{p}}} \\
      v_{f} = \sqrt{\frac{2 \cdot 1.6 \times 10^{-19} \cdot 540}{1.67 \times 10^{-27}}} \\
      v_{f} = 3.217 \times 10^{5} \mathrm{m/s^2}
    \end{align}
\end{enumerate}

\textbf{Suppose that the proton is again at rest at point A.}

\begin{enumerate}[resume]
  \item How much work would be done in moving this proton from point A to point C?
    \textit{Zero work ($W = 0 \mathrm{J}$), because the movement is perpendicular with the electric field.}

  \item How will the electrostatic potential at point C compare to the electrostatic potential at point A?
    \textit{They will be equal.}

  \item What will be the potential difference between point A and point B?
    \textit{$540 \mathrm{V}$, because $\Delta x$ only includes the vertical components that are parallel with the electric field.}

  \item How much work will have to be done on a proton to move it from point C to point B?
    \textit{$8.64 \times 10^{-17} \mathrm{J}$, for the same reasons as above.}
\end{enumerate}

\item Two parallel plates are arranged as shown to the right. The electric field between the plates is uniform and is directed from the positive plate to the negative plate as shown. The electric field strength is $E=60,000 N/C$ ($6 \times 10^{4} \mathrm{N/C}$) and the two plates are $d=6.00cm$ apart. A particule of charge $q=-0.015\mu C$ ($q = -1.5 \times 10^{-8}$) is initially placed at point A.
\begin{enumerate}
  \item How much work would have to be done to move this particle from point A to point B?
    \begin{align}
      W_{\mathrm{A} \to \mathrm{B}} = F_{E} \Delta x_{\mathrm{A} \to \mathrm{B}} \\
      W_{\mathrm{A} \to \mathrm{B}} = Eq \Delta x_{\mathrm{A} \to \mathrm{B}} \\
      W_{\mathrm{A} \to \mathrm{B}} = 6 \times 10^{4} \cdot -1.5 \times 10^{-8} \times 0.04 \\
      W_{\mathrm{A} \to \mathrm{B}} = -3.6 \times 10^{-5}
    \end{align}

  \item What is the potential difference between point A and point B?
    \begin{align}
      \Delta V_{\mathrm{A} \to \mathrm{B}} = \int E \dd r_{\mathrm{A} \to \mathrm{B}} \\
      \Delta V_{\mathrm{A} \to \mathrm{B}} = Er_{\mathrm{A} \to \mathrm{B}}\\
      \Delta V_{\mathrm{A} \to \mathrm{B}} = 6 \times 10^{4} \cdot 0.04 \\
      \Delta V_{\mathrm{A} \to \mathrm{B}} = 2400 \mathrm{V}
    \end{align}

  \item How much work would have to be done in moving this particle from point A to point C? \textit{$-3.6 \times 10^{-5}$, because $\Delta x$ will only include the horizontal components that are parallel with the electric
  field.}

\item What is the potential difference between points B and C? \textit{Zero, because they have the same horizontal positition.}
\end{enumerate}

\textbf{For parts (e) and (f), suppose that another particle (which has a charge of 2.0 $\mu$C) \text{($2 \times 10^{-6}$) C} is placed, initially, on the negative plate. This particle is then moved from the negative plate to the positive plate.}

\begin{enumerate}[resume]
  \item How much work would be done in moving this particle from the negative plate to the positive plate?
    \begin{align}
      W_{E} = \int F_{E} \dd \Delta x \\
      W_{E} = F_{E} \Delta x \\
      W_{E} = Eq \Delta x \\
      W_{E} = 6 \times 10^{4} \cdot 2 \times 10^{-6} \cdot 0.06 \\
      W_{E} = 7.2 \times 10^{-3} \mathrm{J}
    \end{align}

  \item What will be the potential difference between these two plates?
    \begin{align}
      \Delta V = E \Delta x \\
      \Delta V = 6 \times 10^{4} \cdot 0.06 \\
      \Delta V = 3600 \mathrm{V}
    \end{align}
\end{enumerate}

\item A small sphere contains a charge of $q_{1}=5.0\mu C$ ($5 \times 10^{-6}$ C) as shown below.
\begin{enumerate}
  \item What will be the direction and magnitude of the electric field at point A?
    \begin{align}
      E_{\mathrm{A}} = \frac{F_{E}}{q} = \frac{kq}{\Delta x^2} \\
      E_{\mathrm{A}} = \frac{9 \times 10^{9} \cdot 5 \times 10^{-6}}{0.03^{2}} \\
      E_{\mathrm{A}} = 5 \times 10^{7} \mathrm{N/C} \text{ to the right.}
    \end{align}

  \item What will be the direction and magnitude of the electrostatic force acting on a proton placed at point A?
    \begin{align}
      F_{E_{\mathrm{A}}} = \frac{kq_{1}q_{2}}{\Delta x^{2}} \\
      F_{E_{\mathrm{A}}} = \frac{9\times 10^{9} \cdot 5 \times 10^{-6} \cdot 1.6 \times 10^{-19}}{0.03^{2}} \\
      F_{E_{\mathrm{A}}} = 8 \times 10^{-12} \mathrm{N} \text{ to the right.}
    \end{align}

  \item What will be the electrostatic potential at point A?
    \begin{align}
      V_{\mathrm{A}} = E_{\mathrm{A}} \Delta x \\
      V_{\mathrm{A}} = 5 \times 10^{7} \cdot 0.03 \\
      V_{\mathrm{A}} = 1.5 \times 10^{6} \mathrm{V}
    \end{align}

  \item What will be the direction and magnitude of the electric field at point B?
    \begin{align}
      E_{\mathrm{B}} = \frac{F_{E_{\mathrm{B}}}}{q_{2}} = \frac{kq_{1}}{\Delta x_{\mathrm{B}}^2} \\
      E_{\mathrm{B}} = \frac{9 \times 10^{9} \cdot 5 \times 10^{-6}}{0.12^2} \\
      E_{\mathrm{B}} = 3.125 \times 10^{6} \mathrm{N/C} \text{ to the right.}
    \end{align}

  \item What will be the direction and magnitude of the electrostatic force acting on a proton placed at point B?
    \begin{align}
      F_{E_{\mathrm{B}}} = E_{\mathrm{B}} q_{p} \\
      F_{E_{\mathrm{B}}} = 3.125 \times 10^{6} \cdot 1.6 \times 10^{-19} \\
      F_{E_{\mathrm{B}}} = 5 \times 10^{-13} \mathrm{N} \text{ to the right.}
    \end{align}

  \item What will be the electrostatic potential at point B?
    \begin{align}
      V_{\mathrm{B}} = E_{\mathrm{B}} \Delta x_{\mathrm{B}} \\
      V_{\mathrm{B}} = 3.125 \times 10^{6} \cdot 0.12 \\
      V_{\mathrm{B}} = 3.75 \times 10^{5} \mathrm{V}
    \end{align}

  \item What will be the potential difference between points A and B?
    \begin{align}
      \Delta V_{\mathrm{A} \to \mathrm{B}} = V_{\mathrm{A}} - V_{\mathrm{B}} \\
      \Delta V_{\mathrm{A} \to \mathrm{B}} = 1.5 \times 10^{6} - 3.75 \times 10^{5} \\
      \Delta V_{\mathrm{A} \to \mathrm{B}} = 1.125 \times 10^{6} \mathrm{V}
    \end{align}

  \item How much work would be required to move a proton from point B to point A?
    \begin{align}
    \end{align}

  \item How much work would be required to move a proton from point A to point B?
    \begin{align}
      W_{\mathrm{A} \to \mathrm{B}} = -W_{\mathrm{B} \to \mathrm{A}} \\
      W_{\mathrm{A} \to \mathrm{B}} = -1.8 \times 10^{-13} \mathrm{J}
    \end{align}

  \item Which point is at the higher potential, A or B? Explain!
    \textit{Point A, as it is closer to the positive charge.}

  \item What will be the electrostatic potential at infinity?
    \textit{Zero Volts.}

  \item What would be the potential difference between infinity and point B?
    \begin{align}
      \Delta V_{\mathrm{B} \to \infty} = V_{\mathrm{B}} \\
      \Delta V_{\mathrm{B} \to \infty} = 3.75 \times 10^{5} \mathrm{V}
    \end{align}

  \item How much work would be required to move a proton from infinity to point B?
    \begin{align}
      V_{\infty} = F_{E_{\infty}} = E_{\infty} = 0 \\
      \dots \\
      W_{\infty \to \mathrm{B}} = U_{E_{\mathrm{B}}} \\
      W_{\infty \to \mathrm{B}} = V_{\mathrm{B}} q_{p} \\
      W_{\infty \to \mathrm{B}} = 3.75 \times 10^{5} \cdot 1.6 \times 10^{-19} \\
      W_{\infty \to \mathrm{B}} = 6 \times 10^{-14} \mathrm{J} 
    \end{align}

  \item How much work would be required to bring an electron from infinity to point B?
\end{enumerate}

\item An atom of Carbon 12 contains 6 protons in its nucleus.
\begin{enumerate}
  \item What will be the total charge of the nucleus of a Carbon 12 atom?
  \item What will be the strength of the electric field a distance 0.5 $\AA$ [$10^{10}$ Angstrom = 1m] from this
    Carbon nucleus?
  \item What will be the electrostatic potential a distance 0.5 $\AA$ from this carbon nucleus?
  \item What will be the electrostatic potential infinitely far from this Carbon nucleus?
  \item What will be the potential difference between a point 0.5 $\AA$ from the Carbon nucleus and infinity?
  \item How much work will be done in moving an electron from infinity to a point $0.5 \AA$ from the nucleus of the Carbon nucleus
  \item What will be the potential difference between a point 0.5 $\AA$ from the nucleus of a carbon atom and a point 1.5 $\AA$ from that same nucleus?
  \item How much work will be done in moving an electron from a point 0.5 $\AA$ from the nucleus of a Carbon atom to a point 1.5 $\AA$ from the same Carbon nucleus?
\end{enumerate}

\item Protons in the nucleus of an atom are on average a distance of 3.0 Fermi [$10^{-15}$m] apart.
\begin{enumerate}
  \item What will be the electrostatic potential 3.0 Fermi from a proton?
  \item What will be the electrostatic potential infinitely far away from a proton?
  \item What will be the potential difference between a point infinitely far away from a proton and a point 3.0 Fermi from a proton?
  \item How much work would be required to move a proton from infinity to a point 3.0 Fermi from a second proton?
\end{enumerate}

\textbf{Suppose that you hold onto one of these protons and allow the other to accelerate away to infinity.}

\begin{enumerate}[resume]
  \item What will be the velocity of this proton when it is very far away?
\end{enumerate}

\end{enumerate}

{\large PHYS C 143 \\ CALCULATING POTENTIAL VIA INTEGRATION}

\begin{enumerate}

\item Consider two horizontal, parallel plates, each with an area of 3.0$m^2$, separated by a distance of $d=3.5$cm.  These two plates are connected to a battery and, as a result, the upper plate gains a charge of $Q_{1} = 12\mu C$ and the lower plate gains a charge of $Q_{2} = -12\mu C$.
\begin{enumerate}
  \item What will be the strength of the electric field in the area between these two plates?
  \item What will be the strength of the electric field in the area outside of these two plates?
  \item What will be the potential difference between these two plates?
  \item How much work would be done on an electron in moving it from the positive plate to the negative plate?
  \item Suppose that this electron is then released and is allowed to accelerate back to the positive plate, what will be the velocity of the electron just as it reaches the positive plate?
\end{enumerate}

\item Consider two concentric, conducting shells of radii of $R_{1} = 4.0cm$ and $R_{2} = 6.0cm$ as shown in the diagram to the right. The inner shell contains a charge of $Q_{1} = -6.0\mu C$ and the outer shell contains a charge of $Q_{2} = +12.0 \mu C$
\begin{enumerate}
  \item What will be the electrostatic potential of the outer shell?
  \item What will be the potential difference between the inner shell and the outer shell?
  \item What will be the electrostatic potential of the inner shell?
  \item What will be the electrostatic potential at the center of these two shells?
  \item How much work would be required to bring a proton from infinity to the outer shell?
\end{enumerate}

\textbf{Suppose that the radius of the outer shell increases until $R_{2}$ becomes 9.0cm.}

\begin{enumerate}[resume]
  \item What will be the potentials of the outer and inner shells?
  \item What will be the new potential difference between these two shells?
\end{enumerate}

\item Consider two concentric, conducting cylindrical shells which have radii $R_{1} = 4.0cm$ and $R_{2} = 6.0cm$ and are $L= 35.0m$ long [same diagram as above]. The inner shell contains a charge of $Q_{1} = -12.0 \mu C$ and the outer shell contains a charge of $Q_{2} = +12.0 \mu C$
\begin{enumerate}
  \item What will be the electrostatic potential of the outer shell?
  \item What will be the potential difference between the outer shell and the inner shell?
  \item What will be the electrostatic potential of the inner shell?
  \item What will be the electrostatic potential at the center of these two shells?
  \item How much work must be done to transfer a single electron from the outer shell to the inner shell?
  \item How much work would be required to bring a proton from infinity to the outer shell?
\end{enumerate}

\item Consider a conducting spherical shell, which has an inner radius of $R_{1} = 7.00cm$, an outer radius of $R_{2} = 9.00cm$, and contains a charge of $Q_{2} = +18.0\mu C$. This shell, in turn, encloses a point charge of $Q_{1} = -6.0 \mu C$ located at its center as shown in the diagram to the right.
\begin{enumerate}
  \item What will be the electrostatic potential on the ouside of the conducting shell?
  \item What will be the potential difference between the inside of the conducting shell and the outside of the conducting shell?
  \item What will be the electrostatic potential on the inside of the conducting shell?
  \item What will be the electostatic potential at a point 1.0 cm from the central charge?
  \item What will be the electrostatic potential at the location of $Q_{1}$?
  \item On the graph below, sketch the electric field strength as a function of the distance from $Q_{1}$.
  \item On the graph below, sketch the electrostatic potential as a function of the distance from $Q_{1}$.
  \item Where, other than infinity, will the absolute potential be zero?
\end{enumerate}

\item Consider three, vertical, parallel, conducting plates as shown to the right. Each plate has an area of $3.0 m^{2}$. The first plate contains a charge of $Q_{1} = +6.00\mu C$ and is located at $x = -1.00 m$. The second plate contains a charge of $Q_{2} = -6.0\mu C$ and is located at $x = -0.50m$. The third plate contains a charge of $Q_{3} = +3.00\mu C$ and is located at $x = 0.50 m$. Assume that absolute electrostatic potential is zero at the origin.
\begin{enumerate}
  \item Determine the electric field everywhere.
  \item What will be the electrostatic potential of each plate?
  \item What will be the potential difference between plate \#1 and plate \#2?
  \item How much work would be required to move a proton from plate \#1 to plate \#3?
\end{enumerate}

\item What will be the electrostatic potential of a point \textbf{P} which is both 12.0 cm from a $25.0 \mu C$ charge and 6.0 cm from a 50 $\mu C$ charge?

\item Determine the electrostatic potential at point P in each of hte following diagrams.

\item Suppose that in each diagram above, a 7.00 $\mu C$ charge is to be moved from infinity to point X. In each case above, determine how much work would be required to place the 7.00 $\mu C$ charge at point X.

\item What will be the electrostatic potential energy of each of the set of charges above? [including the 7.0 $\mu C$ charge!]

\item Each of the following questions refers to the diagram below.

\begin{enumerate}
  \item At which point in the above diagram will the electric field strength be the greatest?
  \item At which point in the above diagram will the electrostatic potential be the greatest?
  \item At which point in the above diagram will the electric field potential be the weakest?
  \item At which point in the above diagram will the electrostatic potential be the least?
\end{enumerate}

\end{enumerate}

\end{document}
